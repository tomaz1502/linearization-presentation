\documentclass[usepdftitle=false,aspectratio=169,usenames,dvipsnames]{beamer}
\usepackage{amsmath, amsfonts, amssymb}
\usepackage{algorithm}
\usepackage[noend]{algpseudocode}
\usepackage{xspace}
\usepackage{xparse}
\usepackage[yyyymmdd]{datetime}
\usepackage{ulem}
\usepackage{array}
\usepackage{listings}
\usepackage{multirow}
\usepackage{booktabs}
\usepackage{stmaryrd}
\usepackage{subcaption}
\usepackage{bussproofs}
\usepackage{mathpartir}
\usepackage{minted}
% \usemintedstyle{trac}
\usepackage{xcolor}
\usepackage{proof}
\usepackage{tcolorbox}
\usepackage{relsize}

\usepackage{tikz}
\def\lstlanguagefiles{lstlean.tex}
\lstset{language=lean}

\definecolor{keywordcolor}{rgb}{0.7, 0.1, 0.1}   % red
\definecolor{tacticcolor}{rgb}{0.0, 0.1, 0.6}    % blue
\definecolor{commentcolor}{rgb}{0.4, 0.4, 0.4}   % grey
\definecolor{symbolcolor}{rgb}{0.0, 0.1, 0.6}    % blue
\definecolor{sortcolor}{rgb}{0.1, 0.5, 0.1}      % green
\definecolor{attributecolor}{rgb}{0.7, 0.1, 0.1} % red
\usetikzlibrary{shapes,arrows,positioning,decorations.markings,matrix,fit,calc}
\tikzstyle{line} = [draw, -latex']
\tikzstyle{rline} = [draw, latex'-]
\pgfdeclarelayer{background}
\pgfsetlayers{background,main}
\EnableBpAbbreviations

\DeclareUnicodeCharacter{2081}{$_{1}$}
\DeclareUnicodeCharacter{03B1}{$\alpha$}
\DeclareUnicodeCharacter{2200}{$\mathbb{\forall}$}
\DeclareUnicodeCharacter{2203}{$\mathbb{\exists}$}

\DeclareUnicodeCharacter{2265}{$\ge$}
\DeclareUnicodeCharacter{2227}{$\wedge$}
\DeclareUnicodeCharacter{2243}{$\simeq$}
\DeclareUnicodeCharacter{2264}{$\mathbb{\le}$}
\DeclareUnicodeCharacter{2200}{$\mathbb{\forall}$}
\DeclareUnicodeCharacter{2228}{$\vee$}
\DeclareUnicodeCharacter{2084}{$_{4}$}
\DeclareUnicodeCharacter{2083}{$_{3}$}
\DeclareUnicodeCharacter{2082}{$_{2}$}
\DeclareUnicodeCharacter{2081}{$_{1}$}
\DeclareUnicodeCharacter{208A}{$_{+}$}
\DeclareUnicodeCharacter{1D62}{$_{i}$}
\DeclareUnicodeCharacter{208B}{$_{-}$}
\DeclareUnicodeCharacter{03B1}{$\alpha$}

\hypersetup{
  pdftitle={Formalization of a Proof Calculus for Incremental
Linearization for Satisfiability Modulo Nonlinear
Arithmetic and Transcendental Functions}
  pdfauthor={Tomaz Mascarenhas, Harun Khan, Abdalrhman Mohamed, Andrew Reynolds, Haniel Barbosa, Clark Barrett, Cesare Tinelli},
  pdfkeywords={},
  pdfborder={0 0 0}
  draft=false,
  bookmarksnumbered,
  bookmarksdepth=2,
  bookmarksopenlevel=2,
  bookmarksopen
}

\renewcommand{\dateseparator}{--}

%%% BEGIN Defining style
\useinnertheme[outline]{chamfered}
\setbeamertemplate{navigation symbols}{}%remove navigation symbols
\usefonttheme[onlymath]{serif}%beamer math looks like article math
\usecolortheme{seahorse}
\setbeamercolor{palette quinary}{use=structure,fg=black,bg=white}
%%% END Defining style

%%% BEGIN Biblatex for references
\usepackage[backend=bibtex,
doi=false,isbn=false,url=false,
% show initials only
style=alphabetic,
% Show name
% style=authoryear-icomp,
% citation label has 4 initials if up to 4 authors, 3 and "+" otherwise
maxalphanames= 4, maxcitenames = 3, minalphanames = 3, minnames = 3]{biblatex}
% citation label has first author's name et al
% maxcitenames=2,backend=biber]{biblatex}

\bibliography{./refs.bib}

%%%%%%%%%%%%% macros

\mathchardef\mhyphen="2D % Define a "math hyphen"

\usepackage{pifont}

\newcommand{\bluecheck}{{\color{blue}\checkmark}}
\newcommand{\greencheck}{{\color{blue}\checkmark}}
\newcommand{\redmark}{\color{red}{\ding{55}}}

\newcommand\vvthinspace{\kern+0.041667em}
\newcommand\vthinspace{\kern+0.083333em}
\newcommand\negvthinspace{\kern-0.083333em}
\newcommand\negvvthinspace{\kern-0.041667em}

\newcommand\FV[1]{\textrm{FV}(#1)}

\newcommand\sym[1]{\mathsf{#1}}

%ADT font macros
\newcommand{\typename}[1]{\mathbf{\mathrm{#1}}}
\newcommand{\consname}[1]{\mathrm{#1}}
\newcommand{\selname}[1]{\mathrm{#1}}
\newcommand{\testername}[1]{\mathrm{#1}}

\newcommand{\SInt}{\typename{Int}}
\newcommand{\SBool}{\typename{Bool}}
\newcommand{\SStr}{\typename{Str}}

\newcommand{\evalfn}[1]{\mathrm{eval}_{#1}}

\DeclareDocumentCommand{\qnt}{ O{n} m O{\psi}}{\innerqnt(#2,#1) #3}
\def\innerqnt(#1,#2,#3){#1 \bar #2.\>}

\DeclareDocumentCommand{\terms}{ m O{}}{\mathbf{T}_{#2}(#1)}

\DeclareDocumentCommand{\bfapp}{ O{f} m O{n}}{#1({\bar #2}_{#3})}
\DeclareDocumentCommand{\enum}{ O{n} m O{,\,} O{} O{1}}{#2_#5#4#3\dots #3#2_{#1}#4}
\DeclareDocumentCommand{\benum}{ O{n} m O{\mapsto} O{} O{,\,} O{1}}{\innerbenum(#2,#3,#6,#4)#5\dots #5\innerbenum(#2,#3,#1,#4)}
\def\innerbenum(#1,#2,#3,#4,#5){#1_{#4}#3 #5{#2_{#4}}}

\DeclareDocumentCommand{\bfenum}{O{\mapsto} m}{\innerbfenum(#2,#1)}
  \def\innerbfenum(#1,#2,#3){\bar #1#3 \bar #2}
\DeclareDocumentCommand{\fapp}{ O{f} m O{n} O{,\,} O{(} O{)}}{#1#5#2_1#4\dots #4#2_{#3}#6}

% Column that takes width (left justified): L{1cm}, e.g.
\newcolumntype{L}[1]{>{\raggedright\arraybackslash$} p{#1} <{$}}

% Column with math mode in displaystyle
\newcolumntype{F}[1]{>{$\displaystyle} #1 <{$}}

% Column with math mode in displaystyle
\newcolumntype{D}[1]{>{\displaystyle} #1 <{}}

% Allows to break like in cell; first arg is $t,b,c$, for alignment with other cells
\newcommand{\specialcell}[2][c]{%
  \begin{tabular}[#1]{@{}l@{}}#2\end{tabular}}

\newcommand{\ccfv}{\textsc{CCFV}}

% \newcommand{\cdclt}{CDCL($\mathcal{T}$)}
\newcommand{\cdclt}{CDCL(T)}
\newcommand{\dpllt}{\cdclt}

\newcommand{\cvcho}{\textsc{cvc}4-\textsc{ho}\xspace}
\newcommand{\cvc}{\textsc{cvc}4\xspace}
\newcommand{\zzz}{Z3}
\newcommand{\verit}{\textsc{veriT}\xspace}

\newcommand{\basestrat}[1]{{\bf #1}}
\newcommand\interleavestrat[2]{\basestrat{#1}+\basestrat{#2}}
\newcommand\mathinterleavestrat[2]{$\mathbf{#1}$+$\mathbf{#2}$}
\newcommand\prioritystrat[2]{\basestrat{#1};\basestrat{#2}}
\newcommand{\zbasestrat}[1]{{\bf z3 #1}}
\newcommand{\zprioritystrat}[2]{{\bf z3} \basestrat{#1};\basestrat{#2}}
\newcommand{\E}{\mathsf{E}}
\newcommand{\Q}{\mathsf{Q}}
\newcommand{\M}{\mathsf{M}}
\newcommand{\ite}{\mathsf{ite}}
\newcommand{\Mod}{\mathcal{M}}
\newcommand{\termordereq}{\preceq}
\newcommand{\termorder}{\prec}
\newcommand{\teq}{\simeq}
\newcommand{\tneq}{\not\simeq}

\newcommand{\inputform}{\psi}
\newcommand{\matrixform}{\varphi}

\newcommand\vitem{\vfill\item}
\newcommand\pvitem{\pause\vfill\item}
\newcommand\pitem{\pause\item}

\newcommand{\cvcsy}{\textsc{cvc}4\textsc{sy}\xspace}
\newcommand{\cvccegis}{\textsc{cvc+c}\xspace}
\newcommand{\cvcu}{\textsc{cvc+upi}\xspace}
\newcommand{\cvcuci}{\textsc{cvc+upi+e}\xspace}
\newcommand{\cvcport}{\textsc{cvc-port}\xspace}
\newcommand{\loopinv}{\textsc{loopinvgen}\xspace}

\newcommand{\divc}{D\&C\xspace}

% AJR : update these
\newcommand{\unif}{Unif+PI\xspace}
\newcommand{\unifci}{Unif+PI+E\xspace}
\newcommand{\smtclass}{\textsc{SMTClassify}\xspace}
\newcommand{\learn}{\textsc{Learn}\xspace}
\newcommand{\classchecker}{\textsc{ClassChecker}\xspace}

%%%%%%%%%%%%


% To avoid number in slides that break, like the references one
\setbeamertemplate{frametitle continuation}{}

% Small font
\renewcommand*{\bibfont}{\scriptsize}
%%% END Biblatex for references



%%% BEGIN DEFINING TITLE PAGE

\defbeamertemplate{title page}{mystyle}[1][]
{
  \vbox{}
  \vfill
  \begingroup
    \centering
    {\usebeamercolor[fg]{titlegraphic}\inserttitlegraphic\par}
    \vskip1em
    \begin{beamercolorbox}[sep=0pt,center,#1,rounded=true,wd=12cm]{title}
      \usebeamerfont{title}\inserttitle\par%
      \ifx\insertsubtitle\@empty%
      \else%
        \vskip0.25em%
        {\usebeamerfont{subtitle}\usebeamercolor[fg]{subtitle}\insertsubtitle\par}%
      \fi%
    \end{beamercolorbox}%
    \vskip1em\par
    \begin{beamercolorbox}[sep=8pt,center,#1]{author}
      \usebeamerfont{author}\insertauthor
    \end{beamercolorbox}
    \begin{beamercolorbox}[sep=0pt,center,#1]{date}
      \usebeamerfont{date}\insertdate
    \end{beamercolorbox}%
  \endgroup
  \vfill
}

\setbeamertemplate{title page}[mystyle]

%%% END DEFINING TITLE PAGE

%%% BEGIN Meta information
\author[Tomaz Mascarenhas]
{%
  \texorpdfstring{
\textbf{Tomaz Mascarenhas}\footnotemark[1], Harun Khan\footnote{Equal contribution}, Abdalrhman Mohamed, Andrew Reynolds, Haniel Barbosa, Clark Barrett, Cesare Tinelli
      \vspace{3ex}
    \begin{columns}
      \column{.45\linewidth}
        \centering
        \includegraphics[height=0.25\textheight]{images/DCC_LOGO.jpg}
      \column{.35\linewidth}
        \includegraphics[height=0.15\textheight]{images/Logo_UFMG.jpg}
    \end{columns}
  }
  {}
}

\date{Belo Horizonte, Brasil, 11/04/2025}

\title[Formalization of a Proof Calculus for Incremental Linearization for
  NTA Satisfiability]{Formalization of a Proof Calculus for Incremental Linearization for
  NTA Satisfiability}

% Redifining footer
\makeatother
\setbeamertemplate{footline}
{
  \leavevmode%
  \hbox{%
    \begin{beamercolorbox}[wd=.6\paperwidth,ht=2.5ex,dp=1ex,left]{title in head/foot}%
      {\hspace{1em}\usebeamerfont{title in head/foot}\insertshorttitle}
    \end{beamercolorbox}%
    \begin{beamercolorbox}[wd=.2\paperwidth,ht=2.5ex,dp=1ex,left]{title in head/foot}%
      {\hspace{1em}\usebeamerfont{title in head/foot}Tomaz Mascarenhas, UFMG}
    \end{beamercolorbox}%
    \begin{beamercolorbox}[wd=.2\paperwidth,ht=2.5ex,dp=1ex,right]{title in head/foot}%
      \insertframenumber{} / \inserttotalframenumber\hspace{1em}
    \end{beamercolorbox}}%
  \vskip0pt%
}
\makeatletter
%%% END Meta information

%%% BEGIN Tikz setup
\usepackage{tikz}
\usetikzlibrary{positioning,calc,intersections,backgrounds, fadings}
\usetikzlibrary{shapes}

  \tikzset{
    invisible/.style={opacity=0},
    visible on/.style={alt=#1{}{invisible}},
    alt/.code args={<#1>#2#3}{%
      \alt<#1>{\pgfkeysalso{#2}}{\pgfkeysalso{#3}} % \pgfkeysalso doesn't change the path
    },
  }

\definecolor{BloodRed}{rgb}{.86,0,0}
\definecolor{gold(metallic)}{rgb}{0.83, 0.69, 0.22}

\definecolor{procColor}{rgb}{0.628,0.35,0.17}
\definecolor{cnfColor}{rgb}{0.82, 0.0, 0.86}
% \definecolor{cnfColor}{rgb}{.83,.66,0}
\definecolor{satColor}{rgb}{.83,0,0}
\definecolor{thSolverColor}{rgb}{0,.5,0}
\definecolor{thRwColor}{rgb}{0,0,1}
\definecolor{thCombColor}{rgb}{0.5,0.5,0}

% spec
\definecolor{spec}{rgb}{0.0, 0.0, 0.86}
% syntax recstrictions
\definecolor{grammar}{rgb}{0.82, 0.0, 0.86}

\tikzfading[
  name=arrowfading,
  top color=transparent!0,
  bottom color=transparent!95
]
\newcommand*{\tikzarrow}[2]{%
  \tikz[
    baseline=(A.base),             % Set baseline to the baseline of node content
    font=\footnotesize\sffamily    % Set fontsize of the node content
  ]
  \node[
    single arrow,                  % Shape of the node
    single arrow head extend=4pt,  % Actual width of arrow head
    single arrow tip angle=150,    % Adjust arrow tip angle
    shape border rotate=270,       % Rotate the arrow shape to point down
    draw=red!25,                   % Draw the node shape (with certain border color
    inner sep=2pt,                 % Separation between node content and node shape
    top color=white,               % Shading color on top of node
    bottom color=#1,               % Shading color on bottom of node
    general shadow={               % Specifications for the shadow
      fill=black,
      shadow yshift=-0.5ex,
      path fading=arrowfading
    }
  ] (A) {#2};%
}

%%% END Tikz setup

\newcommand{\qgo}[1]{\innerqgo(#1)}
    \def\innerqgo(#1,#2,#3){#1 #2_1\dots #2_n.~{#3}}
\def\qg#1{\innerqg(#1)}
    \def\innerqg(#1,#2,#3){#1\mathbf{#2}.~{#3}}
\DeclareDocumentCommand{\qgb}{ m O{} }{\innerqgb(#1,#2)}
\def\innerqgb(#1,#2,#3,#4){#1\bar #2_{#4}.~{#3}}
\DeclareDocumentCommand{\fapp}{ O{f} m O{n} O{,} O{(} O{)}}{#1#5#2_1#4\dots #4#2_{#3}#6}
\DeclareDocumentCommand{\fappint}{ O{f} m O{n} O{,}}{\innerfappint(#1,#2,#3,#4)}
  \def\innerfappint(#1,#2,#3,#4,#5,#6){#1(#3#2_1#4#6\dots #6#3#2_{#5}#4)}
\DeclareDocumentCommand{\bfapp}{ O{f} m O{}}{#1(\bar {#2}_{#3})}

\DeclareDocumentCommand{\enum}{ O{n} m O{,} O{} O{1}}{#2_#5#4#3\dots #3#2_{#1}#4}
\DeclareDocumentCommand{\benum}{ O{n} m O{\mapsto} O{} O{,\,} O{1}}{\innerbenum(#2,#3,#6,#4)#5\dots #5\innerbenum(#2,#3,#1,#4)}
  \def\innerbenum(#1,#2,#3,#4,#5){#1_{#4}#5#3 {#2_{#4}}#5}

\usepackage{soul,xcolor}

\setstcolor{red}
\makeatletter
\newcommand\SoulColor{%
  \let\set@color\beamerorig@set@color
  \let\reset@color\beamerorig@reset@color}
\makeatother

\DeclareDocumentCommand{\soutthick}{ O{.4pt} m }{%
    \renewcommand{\ULthickness}{#1}%
       \sout{#2}%
    \renewcommand{\ULthickness}{.4pt}% Resetting to ulem default
}

\definecolor{ao}{rgb}{0.0, 0.0, 1.0}
\definecolor{ao(english)}{rgb}{0.0, 0.5, 0.0}
\definecolor{asparagus}{rgb}{0.53, 0.66, 0.42}
\definecolor{bostonuniversityred}{rgb}{0.8, 0.0, 0.0}
\definecolor{darkbrown}{rgb}{0.4, 0.26, 0.13}
\definecolor{alizarin}{rgb}{0.82, 0.1, 0.26}
% 2952bfff
\definecolor{ceruleanblue}{rgb}{0.16, 0.32, 0.75}
% 8c008cff
\definecolor{darkmagenta}{rgb}{0.55, 0.0, 0.55}
% #8FBC8F
\definecolor{darkseagreen}{rgb}{0.56, 0.74, 0.56}

\usepackage{tikz-qtree}

%%%%%%%%%%%%%%%% BEGIN define outline machinery

\newcommand{\beamersec}[2]{%
  \section*{#1}
  \addcontentsline{toc}{section}{$\bullet$ #1 \vspace{10pt}\newline  \includegraphics[height=.3\textheight]{#2}\vfill}

  \begin{frame}
    \vfill
    \begin{center}
      \vfill\textbf{#1}\vfill
      \includegraphics[height=.5\textheight]{#2}
      \vfill
    \end{center}
    \vfill
  \end{frame}
}

\newcommand{\beamersubsec}[1]{%
  \subsection*{#1}
  \addcontentsline{toc}{section}{$\mbox{}\quad-$ #1 \vspace{10pt}\newline\vfill}

  \begin{frame}
    \vfill
    \begin{center}
      \textbf{#1}
    \end{center}
    \vfill
  \end{frame}
}

\newcommand{\beamersubsubsec}[1]{%
  \subsection*{#1}
  \addcontentsline{toc}{section}{$\mbox{}\qquad\cdot$ #1 \vspace{10pt}\newline\vfill}
}

%%%%%%%%%%%%%%%% END define outline machinery

%%% BEGIN trick for not counting references

\newcommand{\backupbegin}{
   \newcounter{framenumberappendix}
   \setcounter{framenumberappendix}{\value{framenumber}}
}
\newcommand{\backupend}{
   \addtocounter{framenumberappendix}{-\value{framenumber}}
   \addtocounter{framenumber}{\value{framenumberappendix}}
}

%%% END trick for not counting references

\newcommand{\red}[1]{\textcolor{red}{#1}}

% Creating boxes with width of given text; first parameter (optional) is
% alignment of box, then text to be taken the width, then the text to be boxed
\newlength\stextwidth
\newcommand\makesamewidth[3][c]{%
  \settowidth{\stextwidth}{#2}%
  \makebox[\stextwidth][#1]{#3}%
}

% color highlight
% f0dc9cff

\AtBeginSection[]{
  {
    \setbeamertemplate{footline}{}
  \begin{frame}
  \vfill
  \centering
  \begin{beamercolorbox}[sep=8pt,center,shadow=true,rounded=true]{title}
    \usebeamerfont{title}\insertsectionhead\par%
  \end{beamercolorbox}
  \vfill
\end{frame}
}
\addtocounter{framenumber}{-1}
}

% Theorem settings
\addtobeamertemplate{theorem begin}{%
  \setbeamercolor{block title}{fg=white,bg=black}%
  \setbeamercolor{block body}{fg=black, bg=white!90!black}%
}{}

%%% BEGIN CHANGING BULLETS STYLE

\setbeamertemplate{itemize item}{$\vartriangleright$}
\setbeamertemplate{itemize subitem}{$\blacktriangleright$}
\newlength\origleftmargini
\setlength\origleftmargini\leftmargini
\newlength\origleftmarginii
\setlength\origleftmarginii\leftmarginii
\setlength\leftmargini{1em}
\setlength\leftmarginii{1.25em}
\setlength\leftmarginiii{1em}

%%% END CHANGING BULLETS STYLE

\setbeamersize{text margin left=1em, text margin right=1.5em}

\newcommand{\yell}[1]{{\textcolor{red}{[#1]}}}

\newcommand{\cvcc}{\textsc{cvc}{\small 4}\textsc{sy}\textsc{+b}\xspace}
\newcommand{\cvca}{\textsc{cvc}{\small 4}\textsc{sy}\textsc{+u}\xspace}
\newcommand{\gpid}{\textsc{GPiD}\xspace}
\newcommand{\explain}{\textsc{Explain}\xspace}

\newcommand{\mysetminusD}{\hbox{\tikz{\draw[line width=0.6pt,line cap=round] (3pt,0) -- (0,6pt);}}}
\newcommand{\mysetminusT}{\mysetminusD}
\newcommand{\mysetminusS}{\hbox{\tikz{\draw[line width=0.45pt,line cap=round] (2pt,0) -- (0,4pt);}}}
\newcommand{\mysetminusSS}{\hbox{\tikz{\draw[line width=0.4pt,line cap=round] (1.5pt,0) -- (0,3pt);}}}

\newcommand{\mysetminus}{\mathbin{\mathchoice{\mysetminusD}{\mysetminusT}{\mysetminusS}{\mysetminusSS}}}

\def\vv{\smallskip}

\newcommand{\sep}{~\mid~}

\makeatletter
\newcommand\notsotiny{\@setfontsize\notsotiny{7}{8}}
\makeatother

\newcommand\chighlight[2]{\setlength{\fboxsep}{0pt}\colorbox{#1}{#2\strut}}

\begin{document}

{
\setbeamertemplate{footline}{}
\frame{\titlepage}
}
\addtocounter{framenumber}{-1}

\begin{frame}
  \frametitle{Aritmética Não-linear e Funções Transcendentais}
  \begin{itemize}
    \item A teoria NTA em SMT compreende aritmética não-linear e algumas funções transcendentais
          \begin{itemize}
                  \item Funções trigonométricas, $\exp$ e $\log$.
          \end{itemize}
    \vitem Aplicações em: verificação formal de sistemas dinâmicos, processamento de sinais digitais e planejamento de movimentação para robôs.
          \begin{itemize}
                  \item Automatização em assistentes de demonstração, se for provido o ferramental necessário
          \end{itemize}
    \vitem Complexidade $\mathcal{O}(2^{2^{n}})$ para aritmética não-linear em $\mathbb{R}$ e indecidível para NTA.
  \end{itemize}
\end{frame}

\begin{frame}
  \frametitle{Incremental Linearization}
  \begin{itemize}
    \item \textit{Incremental linearization} é um método para decidir a satisfatibilidade de fórmulas em NTA.
          \begin{itemize}
                  \item Não é um procedimento de decisão
          \end{itemize}
    \vitem Dada uma fórmula $\psi$ em NTA, abstraem-se todas as multiplicações e funções transcendentais usando funções não-interpretadas,
    obtendo-se uma fórmula $\psi'$ em UFLRA.
    \vitem Se $\psi'$ não é satisfatível, então $\psi$ também não é.
  \end{itemize}
\end{frame}

\begin{frame}
  \frametitle{Incremental Linearization}
  \begin{itemize}
    \item Se $\psi'$ é satisfatível, não necessariamente $\psi$ também é. Nesse caso, o método introduz incrementalmente novos lemas a $\psi'$ bloqueando soluções espúrias.
    \vfill
\begin{tcolorbox}[colback=green!5,colframe=green!40!black,title=Exemplo]
  Seja $\psi = x^{2} + y^{2} \le 2 \wedge (x \le -1.1 \lor x \ge 1.1) \wedge (y \le -1.1 \lor y \ge 1.1)$.\\
  $\psi'$ seria o mesmo, mas em vez de $x^{2}$ e $y^{2}$ teríamos $f_{*}(x, x)$ e $f_{*}(y, y)$.\\
  É possível interpretar $f_{*}$ tal que $\psi'$ seja satisfeita, mas a fórmula original não é satisfatível.
\end{tcolorbox}
  \end{itemize}
\end{frame}

\begin{frame}
  \frametitle{Incremental Linearization}
  \begin{itemize}
  \item Soluções espúrias envolvendo multiplicações de variáveis são bloqueadas usando a \textit{equação do plano tangente} e o sinal das variáveis.

  \vitem Soluções espúrias envolvendo funções transcendentais são bloqueadas usando o \textit{polinômio de Taylor}.
          \begin{itemize}
                  \item É a ferramenta perfeita para aproximar incrementalmente uma função derivável
          \end{itemize}

  \centering
  \includegraphics[scale=0.1]{images/tanPlane.png}
  \end{itemize}
\end{frame}

\begin{frame}
  \frametitle{Incremental Linearization}
  \begin{algorithmic}[1]
    \Function{IncrementalLinearization}{$\psi$}
      \State $\psi' \gets$ \Call{InitialAbstraction}{$\psi$}
      \State $\Gamma \gets \varnothing$
      \While{\Call{InTimeLimit}{\null}}
        \State $\langle sat, \mu \rangle \gets $\Call{SolveUFLRA}{$\psi' \wedge \Gamma$}
        \If{\textbf{not} $sat$}
          \Return UNSAT
        \EndIf
        \State $\langle sat', \Gamma' \rangle \gets $\Call{CheckRefine}{$\psi', \mu$}
        \If{$sat'$}
          \Return SAT
        \EndIf
        \State $\Gamma \gets \Gamma \cup \Gamma'$
      \EndWhile
    \EndFunction
  \end{algorithmic}
\end{frame}

\begin{frame}
  \frametitle{Proof Calculus}
  \begin{itemize}
    \item cvc5 implementa uma versão modificada desse algoritmo, apropriada para suas otimizações internas.
    \vitem O solucionador define um \textit{proof calculus} composto por 21 lemas que capturam raciocínio por trás do algoritmo, possibilitando a produção de demonstrações.
    \vitem Esses lemas são, em sua grande maioria, as regras usadas para eliminar soluções espúrias.
  \end{itemize}
\end{frame}

\begin{frame}
  \frametitle{Contribuições}
  \begin{itemize}
    \item Neste trabalho nós mecanizamos demonstrações de todas os lemas presentes no proof calculus, fortalecendo a confiança no algoritmo
    \vitem Nesse processo, identificamos algumas hipóteses faltando na documentação de algumas regras, além de outras imprecisões.
    \vitem Além disso, é um primeiro passo em direção a reconstrução de demonstrações em Lean.

  \end{itemize}
\end{frame}

\section{Lemas Formalizados}


\begin{frame}
  \frametitle{Lemas para multiplicação - ARITH\_MULT\_TANGENT}

\begin{center}
  $\infer[]{xy \le bx + ay - ab \leftrightarrow ((x \le a \land y \ge b) \lor (x \ge a \land y \le b))}{- \mid x, y, a, b}$
\end{center}

\begin{center}
  $\infer[]{xy \ge bx + ay - ab \leftrightarrow ((x \le a \land y \le b) \lor (x \ge a \land y \ge b))}{- \mid x, y, a, b}$
\end{center}

\end{frame}

\begin{frame}[fragile]
  \frametitle{Lemas para multiplicação - ARITH\_MULT\_TANGENT}

  \begin{lstlisting}
  theorem arithMulTangentLower {α : Type} [LinearOrderedField α] (x y a b : α) :
      x * y ≤ b * x + a * y - a * b ↔ ((x ≤ a ∧ y ≥ b) ∨ (x ≥ a ∧ y ≤ b))
  \end{lstlisting}

  \begin{lstlisting}
  theorem arithMulTangentUpper {α : Type} [LinearOrderedField α] (x y a b : α) :
      x * y ≥ b * x + a * y - a * b ↔ ((x ≤ a ∧ y ≤ b) ∨ (x ≥ a ∧ y ≥ b))
  \end{lstlisting}

\end{frame}

\begin{frame}
  \frametitle{Lemas para multiplicação - ARITH\_MULT\_SIGN}

\begin{center}
  $\infer[]{(f_{1} \land \cdots \land f_{k}) \rightarrow m \bowtie 0}{- \mid f_{1}, \cdots, f_{k}, m}$
\end{center}
Cada $f_{i}$ é da forma $x_{i} \bowtie 0$. $m$ é um monômio composto por potências dos $x_{i}$'s. A conclusão é definida com base na paridade dos expoentes e no sinal de cada $x_{i}$.
\end{frame}

\begin{frame}[fragile]
  \frametitle{Lemas para multiplicação - ARITH\_MULT\_SIGN}

  \begin{lstlisting}
    theorem powNegOdd      : ∀ {k : ℕ} {r : ℝ}, r < 0 → Odd k  → r ^ k < 0
    theorem powNegEven     : ∀ {k : ℕ} {r : ℝ}, r < 0 → Even k → r ^ k > 0
    theorem nonZeroEvenPow : ∀ {k : ℕ} {r : ℝ}, r ≠ 0 → Even k → r ^ k > 0
    theorem powPos         : ∀ {k : ℕ} {r : ℝ}, r > 0           → r ^ k > 0
  \end{lstlisting}

  \begin{lstlisting}
    theorem combineSigns₁ : ∀ {a b : ℝ}, a > 0 → b > 0 → b * a > 0
    theorem combineSigns₂ : ∀ {a b : ℝ}, a > 0 → b < 0 → b * a < 0
    theorem combineSigns₃ : ∀ {a b : ℝ}, a < 0 → b > 0 → b * a < 0
    theorem combineSigns₄ : ∀ {a b : ℝ}, a < 0 → b < 0 → b * a > 0
  \end{lstlisting}
\end{frame}

\begin{frame}[fragile]
  \frametitle{Lemas para funções transcendentais - Limitando $\exp$ e $\sin$}
\begin{center}
$\infer[]{t \geq c \rightarrow \exp(t) \geq \texttt{maclaurin}(\exp, d, c)}{- \mid d,c,t}$
\end{center}

Onde $d$ é um número ímpar.

\begin{center}
  $\infer[]{(t \geq l \wedge t \le u) \rightarrow
  \sin(t) \geq \texttt{secant}(\sin, l, u, t)}{- \mid d,t,l,u}$
\end{center}

Onde $d$ é congruente a $3$ módulo $4^{*}$, $\sin$ é côncavo no intervalo $[l, u]^{*}$ e \texttt{secant} é a reta secante ao $d$-ésimo polinômio de taylor nos pontos $(l, p(l))$ e $(r, p(r))$.

\end{frame}

\begin{frame}
  \frametitle{Termo complementar de Lagrange}
  \begin{itemize}
    \item Esses resultados seguem do \textit{Teorema de Taylor com termo complementar de Lagrange}:

    \vfill
    \begin{center}
      $f(x) - \left( \mathlarger{\sum_{j=0}^{n}}\dfrac{f^{(j)}(x_0)}{j!}(x - x_0)^j \right) = \dfrac{f^{(n+1)}(x')}{(n+1)!}(x - x_0)^{n+1}$
    \end{center}
    \vitem Não seria possível formalizar esses resultados sem a Mathlib. O resultado acima existe nela, mas assumindo que $x_{0} < x$. Nós estendemos a biblioteca com a versão geral do teorema.
  \end{itemize}
\end{frame}

\begin{frame}[fragile]
  \frametitle{Lemas para funções transcendentais - Limitando $\exp$ e $\sin$}
\begin{lstlisting}
theorem arithTransExpApproxBelow (d n : ℕ) (_ : d = 2*n + 1) :
  Real.exp x ≥ taylorWithinEval Real.exp d Set.univ 0 x



theorem arithTransSineApproxBelowPos (d k : ℕ) (hd : d = 4 * k + 3) (t l u : ℝ) (ht : l ≤ t ∧ t ≤ u)
    (hl : 0 < l) (hu : u ≤ Real.pi) :
    let p : ℝ → ℝ := taylorWithinEval Real.sin d Set.univ 0
    sin t ≥ ((p l - p u) / (l - u)) * (t - l) + p l
\end{lstlisting}
\end{frame}

\begin{frame}
  \frametitle{Outras funções transcendentais}
  \begin{itemize}
    \item $\cos(x) = \sin(x + \frac{\pi}{2})$
    \vitem $\tan(x) = \frac{\sin(x)}{\cos(x)}$
    \vitem $\sec(x) = \frac{1}{\cos(x)}$
    \vitem $\csc(x) = \frac{1}{\sin(x)}$
    \vitem $\cot(x) = \frac{1}{\tan(x)}$
    \vitem $\log$ e trigonométricas inversas são modeladas usando novas variáveis e asserções. Por exemplo, para um termo com a forma $\log(1 + x)$ introduzimos uma variável $y$ e adicionamos a asserção $\exp(y) = 1 + x$.
  \end{itemize}
\end{frame}

\begin{frame}
  \frametitle{Trabalhos futuros}
  \begin{itemize}
    \item Correções no cvc5
    \vitem Reconstrução de demonstrações usando \textit{lean-smt}
    \vitem Desafios com números reais
  \end{itemize}
\end{frame}

\end{document}

%%% Local Variables:
%%% mode: latex
%%% TeX-master: t
%%% End:
